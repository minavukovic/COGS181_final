\documentclass[conference,compsoc]{IEEEtran}
\ifCLASSOPTIONcompsoc
  % IEEE Computer Society needs nocompress option
  % requires cite.sty v4.0 or later (November 2003)
  \usepackage[nocompress]{cite}
\else
  % normal IEEE
  \usepackage{cite}
\fi
\usepackage{booktabs} 
\usepackage{parskip}
\usepackage{enumitem}
\usepackage{multirow}
\usepackage[table,xcdraw]{xcolor}
\usepackage[utf8]{inputenc}
\usepackage{hyperref}
\usepackage{graphicx}
\usepackage[tight,footnotesize]{subfigure}
\DeclareUnicodeCharacter{2212}{-}

\hyphenation{op-tical net-works semi-conduc-tor}


\begin{document}
%
% paper title
% Titles are generally capitalized except for words such as a, an, and, as,
% at, but, by, for, in, nor, of, on, or, the, to and up, which are usually
% not capitalized unless they are the first or last word of the title.
% Linebreaks \\ can be used within to get better formatting as desired.
% Do not put math or special symbols in the title.
\title{Classifying COVID-19 X-rays using Convolutional Neural Networks}


% author names and affiliations
\author{\IEEEauthorblockN{Mina Vukovic \& Prabesh Pokharel}
\IEEEauthorblockA{COGS 181}}

% make the title area
\maketitle

\begin{abstract}
A defining part of 2020 is COVID-19 and how it has rapidly spread across the whole world.
Due to its infectious nature, it is important to be able to detect its presence in someone as quickly as
posible to prevent spreading it further. Here we levereged pretrained models such as AlexNet [4] and ResNet-152 [5]
to speed up the training process and train our own convolutional networks on Chest X-ray images of patients
with COVID-19, pneumonia and no disease. We explore the effectiveness of using deep learning to clasify patients
as whether or not they have coronavirus as an alternative to the current swab method of detection.
The overall accuracy was FILL IN HERE.
\end{abstract}

\IEEEpeerreviewmaketitle

\section{Introduction}
One of the most important ways to help contain a viral disease is through limiting its ability to spread. 
In doing so, it is vital to be able to quickly classify if someone is sick or not with high precision and recall so that it is not passed on further. 
The current method of testing is through a swab that is inserted through the nose, and is then held there for a few seconds in order 
to collect the secretions [1]. Naturally, this test is rather invasive and can be uncomfortable for the patients being tested, 
in addition to the results not being instantaneous and needing to be analysed in a lab. They can also have up to a $30\%$ false negative
rate [2], although the most common test distributed by Roche achieves around a $95\%$ accuracy [3]. Therefore it is useful to explore 
other means of detection such as training convolutional neural networks on chest X-ray imagery which can increase the speed and
potentially accuracy of COVID-19 detection.

Traditionally we train an entire convolutional neural network from scratch i.e. we randomize the initial parameters. 
This process takes a lot of data and time. Due to the limitation of covid chest X-ray data and deep learning architecture 
being such a data hungry model we adapted to use transfer learning techniques to increase the performance of models as well 
as substantially decrease the training time. Transfer learning is the process where we transfer the knowledge gained from one 
task to another task. It usually works best when we take a model trained on a large dataset and transfer its knowledge to a smaller dataset.
There are many pretrained models to choose from, but for our task we choosed AlexNet and ResNet-152 pretrained models and there are 
many transfer learning strategies as well but we are using off-the-shelf pre-trained models as Feature Extractors [6]. 
Our pretrained AlexNet and ResNet-152 models were trained on ImageNet, with millions of images and thousands of classes
[7]. 
We took these pre-trained models and removed the fully connected layer and created our own fully connected neural network tailored 
towards our classification task. We treated the rest of the convolution layer of both architecture as a fixed feature extractor.  


\section{Method}

\subsection{Neural Networks}

\begin{description}[leftmargin=*,labelindent=0pt]
		\item[AlexNet:] 
		
		\item[ResNet-50:] 

		\item[VGG-19:]

		\item[Our Convolution Neural Network:]

\end{description}


\subsection{Data}
Due to the novelty of the disease, there is not as much data widely available and research is still relatively in its infancy. 
Thus any dataset that currently exists for public use is limited. The data that we used originates from 5 different sources, 
which was then combined using this script:
\href{https://github.com/lindawangg/COVID-Net/blob/master/docs/COVIDx.md}{https://github.com/lindawangg/COVID-Net/blob/master/docs/COVIDx.md}. The data in 
this script contains different chest X-Ray datasets that are COVID-19, pneumonia and normal X-rays (as seen in
\textit{Figure 2} for example). The script combines all of 
these different images into one large test and train dataset. In total, there were 470 COVID-19 images, 6050 pneumonia images and 
8851 normal images, for a total of 15,371 images. This is an unbalanced dataset because there were so few COVID-19 images compared to 
the others, which would lead to unoptimised classification performance for the neural networks. 

\begin{figure}
    \centering
    \subfigure[]{\includegraphics[width=0.13\textwidth]{/home/mina/school/cogs181/finalproj/final_data/test/COVID-19/61bc50d1.jpg}} 
    \subfigure[]{\includegraphics[width=0.13\textwidth]{/home/mina/school/cogs181/finalproj/final_data/test/pneumonia/0a51f668-b7b1-4d8d-9ab9-de1f702f071a.png}} 
    \subfigure[]{\includegraphics[width=0.13\textwidth]{/home/mina/school/cogs181/finalproj/final_data/test/normal/0a55219b-5303-4490-b173-6039f355db2f.png}}
    \caption{(a) COVID-19 (b) Pneumonia (c) Normal}
    \label{fig:fig2}
\end{figure}

First we split the data into a shuffled 80:20 train and test split for each class. Then we chose to solve this problem of 
limited training covid data compared to the rest by augmenting the current training COVID-19 data. For each COVID-19 image, 
we applied a rotation (45 degrees, 30 degrees and 72 degrees), shift (by 25x25 pixels and 30x30 pixels), horizontal and vertical flip, 
added noise (sigma=0.1, 0.2 and 0.3) and blurred the image (by factor of 2.5 and 3.5). This allowed for some variety on the 
limited images we had to train the models on different orientations or appearances of the images. 

Next, since there were still more normal images in comparison to the number of training COVID-19 and pneumonia images we now had, we 
undersampled this class to be 5000 samples. As seen in \textit{Table 1}, this finally resulted in the number of images for training being 4888 for COVID-19, 4841 for 
pneumonia and 5000 for normal. The test data was 94 for COVID-19, 1211 for pneumonia and 1771 for normal.
\begin{figure}
    \centering
    \subfigure[]{\includegraphics[width=0.13\textwidth]{/home/mina/school/cogs181/finalproj/final_data/train/COVID-19/rotate45_covid-19-pneumonia-35-1.jpg}} 
    \subfigure[]{\includegraphics[width=0.13\textwidth]{/home/mina/school/cogs181/finalproj/final_data/train/COVID-19/noise03_COVID-19(116).png}} 
    \subfigure[]{\includegraphics[width=0.13\textwidth]{/home/mina/school/cogs181/finalproj/final_data/train/COVID-19/blur4_2168a917.jpg}}
    \caption{Some data augmentations: (a) Rotation (b) Noise (c) Blur}
    \label{fig:fig2}
\end{figure}

\end{description}



\begin{table}[ht]
		\caption{Number of images in each class for the train and test split.} 
		\label{datasets-table} 
		\vskip 0in
		\begin{center}
				\begin{small}
						\begin{sc}
								\begin{tabular}{lcccr}
										\toprule
										\textbf{Class}  & \textbf{Train} & \textbf{Test}\\
										\midrule
										\textbf{Covid-19} & $$4,888$$ & $$94$$ \\
										\textbf{Pneumonia} & $$4,841$$ & $$1,211$$ \\
										\textbf{Normal} & $$5,000$$ & $$1,771$$ \\
										\bottomrule
								\end{tabular}
						\end{sc}
				\end{small}
		\end{center}
		\vskip -0.1in
\end{table}



\section{Experiment}
\begin{figure*}
  \includegraphics[width=\textwidth,height=5cm]{conv.png}
  \caption{Our convolutional network.}
\end{figure*}


\section{Results \& Analysis}
\begin{table}[ht]
\caption{Accuracy of all networks per class and overall.}
\begin{tabular}{c|cccc}
                     & \textbf{COVID-19} & \textbf{Normal} & \textbf{Pneumonia} & \textbf{Overall} \\ \hline
\textbf{Our ConvNet} & 74.16\%           & 90.78\%         & 84.46\%            & 87.00\%                   \\
\textbf{AlexNet}     & 76.04\%           & 92.00\%         & 83.82\%            & 88.00\%                   \\
\textbf{VGG-19-bn}   & 43.52\%           & 92..83\%        & 80.73\%            & 84.00\%                   \\
\textbf{ResNet-50}   & 20.15\%           & 86.45\%         & 78.28\%            & 74.00\%                  
\end{tabular}
\end{table}

\begin{figure*}
  \includegraphics[width=\textwidth,height=9cm]{traininglosscurve.png}
  \caption{Training loss curve of all the neural networks.}
\end{figure*}


\section{Conclusion}




\begin{thebibliography}{1}

\bibitem{IEEEhowto:kopka}
H.~Kopka and P.~W. Daly, \emph{A Guide to \LaTeX}, 3rd~ed.\hskip 1em plus
  0.5em minus 0.4em\relax Harlow, England: Addison-Wesley, 1999.

\bibitem{Random Forest Classifier}
Pedregosa \emph{et al.}, \emph{Scikit-learn: Machine Learning in Python}. JMLR $12$, pp. $2825-2830$, 2011.

\bibitem{SVM} 
S.~Ray, \emph{Understanding Support Vector Machine algorithm from examples (along with code)}.
Analytics Vidhya, 2017.
\\https://www.analyticsvidhya.com/blog/2017/09/understaing-support-\\vector-machine-example-code/

\bibitem{CNM Paper}
R.~Caruana and A.~Niculescu-Mizil, \emph{An Emperical Comparison of Supervised Learning
Algorithms}. ICML Proceedings of the $23$rd international conference on Machine learning, pp.
$161-168$, $2006$.

\bibitem{Overfitting}
\emph{Overfitting in Machine Learning: What It Is and How to Prevent It}. Elite Data Science, 2017.
https://elitedatascience.com/overfitting-in-machine-learning

\bibitem{Sample Size} 
V.~Popovici \emph{et al.}, \emph{Effect of training-sample size and classification difficulty on
the accuracy of genomic predictors}. Breast Cancer Research, $2010$. doi $10.1186/$bcr$2468$

\bibitem{Sample Size 2} 
J.~Prusa, T.~Khoshgoftaar and N.~Seliya, \emph{The Effect of Dataset Size on Training Tweet
Sentiment Classifiers}. IEEE International Conference on Machine Learning and Applications, $14$th
ed, 2015. doi $10.1109/$ICMLA$.2015.22$

\end{thebibliography}



% that's all folks
\end{document}

 
